%% documentation.tex
%% Copyright 2020 Christian Matt
%
% This work may be distributed and/or modified under the
% conditions of the LaTeX Project Public License, either version 1.3
% of this license or (at your option) any later version.
% The latest version of this license is in
%   http://www.latex-project.org/lppl.txt
% and version 1.3 or later is part of all distributions of LaTeX
% version 2005/12/01 or later.
%
% This work has the LPPL maintenance status `maintained'.
% 
% The Current Maintainer of this work is Christian Matt.
%
% This work consists of the files algpseudocodex.sty and documentation.tex.

\documentclass[11pt,a4paper,USenglish]{article}
\usepackage[T1]{fontenc}
\usepackage{babel}
\usepackage[utf8]{inputenc}
\usepackage[margin=1.0in]{geometry}
\usepackage{microtype}
\usepackage{xcolor}

\usepackage[colorlinks,allcolors=blue!70!black]{hyperref}
\usepackage[capitalise,nameinlink,noabbrev,compress]{cleveref}

\title{\bf{Algpseudocodex Package Documentation}}
\author{Christian Matt \\ \url{https://github.com/chrmatt/algpseudocodex}}

\begin{document}

\maketitle

\begin{abstract}
	This package allows typesetting pseudocode in \LaTeX. It is based on \texttt{algpseudocode} from the \texttt{algorithmicx} package and uses the same syntax, but adds several new features and improvements. Notable features include customizable indent guide lines and the ability to draw boxes around parts of the code for highlighting differences. This package also has better support for long code lines spanning several lines and improved comments.
\end{abstract}

\section{Introduction}
Todo


\end{document}
